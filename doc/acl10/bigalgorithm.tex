In this section, we present our overall algorithm addressing
the problem of identifying taxonomic relation of concepts.

Inputs of our algorithm are a pair of target concepts ($X$, $Y$) and a
trained taxonomic relation classifier $\mathcal{R}$, which is used to
provide a distribution of confidences over all possible taxonomic
relations between any two concepts. The algorithm's output is the
taxonomic relation of $X$ and $Y$, which can be (1) $X$ is an {\em
  ancestor} of $Y$, denoted by $X \leftarrow Y$; (2) $X$ is a {\em
  child} of $Y$, denoted by $X \rightarrow Y$; (3) $X$ and $Y$ are
{\em siblings}, denoted by $X \leftrightarrow Y$; or (4) $X$ and $Y$
have {\em no relation}, denoted by $X \nleftrightarrow Y$.

%%%%%%%%%%%%%%%%%%%%%%%%%%%%%%%%%%%%

\ignore{Input of our problem is a pair of concepts ($X$, $Y$), and output is
their taxonomic relation which can be (1) $X$ is an {\em ancestor} of
$Y$, denoted by $X \leftarrow Y$; (2) $X$ is a {\em child} of $Y$,
denoted by $X \rightarrow Y$; (3) $X$ and $Y$ are {\em siblings},
denoted by $X \leftrightarrow Y$; or (4) $X$ and $Y$ have {\em no
  relation}, denoted by $X \nleftrightarrow Y$.}

\ignore{
\begin{enumerate}
\item $X$ is an ancestor of $Y$, denoted by $X \leftarrow Y$.
\item $X$ is a child of $Y$, denoted by $X \rightarrow Y$.
\item $X$ and $Y$ are siblings, denoted by $X \leftrightarrow Y$.
\item $X$ and $Y$ have no relation, denoted by $X \nleftrightarrow Y$.
\end{enumerate}
}

\ignore{ Our approach consists of two key components: (1) a machine
  learning-based algorithm to classify taxonomic relations; and (2) a
  constraint-based inference model to make final decision using
  relational constraints of taxonomic relations.  }

\ignore{
\begin{enumerate}
\item A machine learning-based algorithm to classify concept relations.
\item A constraint-based inference model to make final decision using
  relational constraints enforced among concept relations.
\end{enumerate}
}

Figure \ref{fig:rel-know-iden-alg} presents our overall algorithm to
identify taxonomic relation of $X$ and $Y$. The algorithm consists of
two main components (1) a machine learning-based algorithm to learn a
local taxonomic relation classifier; and (2) a constraint-based
inference model that makes use of related concepts extracted for $X$
and $Y$ to make final decision on taxonomic relation of X and Y.  Both
components use the same Wikipedia index as a knowledge source to
extract features of input concepts and related concepts to learn the
local classifier and do inference as well.

The identification component (B) makes prediction on the taxonomic
relation between $X$ and $Y$ using $\mathcal{R}$ as its local
classifier.  Given two concepts $X$, and $Y$, function
\texttt{isNotWikipediaConcept} determines if they are {\em Wikipedia
  concepts} or {\em non-Wikipedia concepts} by searching the concept
space in Wikipedia. If a concept is {\em non-Wikipedia} (i.e. it does
not have a Wikipedia page), the algorithm tries to find a replacement
for it by performing a web search via function
\texttt{findReplacement}. Replacing concepts are expected to be {\em
  Wikipedia concepts} and in the same semantic class with the input
concepts. After that, function \texttt{extractRelatedConcepts} returns
list $\mathcal{Z}$ of additional concepts which are related to $X$ and
$Y$. The algorithm passes the two input concepts, local classifier
$\mathcal{R}$ and list $\mathcal{Z}$ to function \texttt{doInference},
which solves a constraint-based optimization problem and returns
$\ell^*$, the taxonomic relation of $X$ and $Y$.

\ignore{We then use two input concepts (or their
  replacements) to build a list of Wikipedia articles that represents
  each input. Following, a machine learning-based classifier is used
  to identify the taxonomic relations between two concepts. Finally,
  we make a final decision using an inference model within a
  constraint-based framework that explores relational constraints
  enforced among concepts involved.}

\begin{figure}[!t]
  \begin{centering}
    {\scriptsize
      \fbox{
        \begin{minipage}{6in} 
          \begin{tabbing}
            {\textsc{A. Learning local classifier $\mathcal{R}$}~~~~~~~~~~~~~~~~~}\\
            \qquad {\textsc{Input}}: ~~~Training set  \\
            \qquad \qquad \qquad \qquad $\mathcal{D}$ = \{taxonomic-relation-annotated concept pairs\} \\
            \qquad \qquad \qquad Wikipedia index $\mathcal{W}$ \\
            \qquad {\textsc{Output}}: A local taxonomic relation classifier $\mathcal{R}$ \\
            \\
            \qquad 1. ~~~~$\mathcal{R} \leftarrow$ \texttt{train($\mathcal{D}$, $\mathcal{W}$)} \\
            \\
            {\textsc{B. On-the-fly Taxonomic Relation Identification}~~~~~~~~~~~~~~~~~~~~~~}\\
            \qquad {\textsc{Input}}: ~~~A concept pair ($X$, $Y$) \\
            \qquad \qquad \qquad A trained taxonomic relation classifier \\
            \qquad \qquad \qquad \qquad $\mathcal{R}$ used to make local prediction. \\
            \qquad \qquad \qquad Wikipedia index $\mathcal{W}$ \\
            \qquad {\textsc{Output}}: Taxonomic relation $\ell^*$ of $X$ and $Y$ \\
            \\
            \qquad 1. ~~~~If \texttt{isNotWikipediaConcept($X$,  $\mathcal{W}$)}, then \\
            \qquad \qquad \qquad $X \leftarrow \texttt{findReplacement}(X, Y)$ \\
            %\qquad End if \\
            \qquad \qquad If \texttt{isNotWikipediaConcept($X$,  $\mathcal{W}$)}, then \\
            \qquad \qquad \qquad $Y \leftarrow \texttt{findReplacement}(Y, X)$ \\
            %\qquad End if \\
            \\     
            \qquad 2. ~~~$\mathcal{Z} \leftarrow \texttt{extractRelatedConcepts}(X,Y)$ \\
            \\
            \qquad 3. ~~~$\ell^* = \texttt{doInference}(X,Y,\mathcal{Z},\mathcal{R},\mathcal{W})$ \\
            \\
            \qquad \textsc{Return}: $\ell^*$; \\
          \end{tabbing}
        \end{minipage}
      }
  }
\end{centering}
\caption{On-the-fly taxonomic relation identification algorithm.}
\label{fig:rel-know-iden-alg}
\end{figure}

\ignore{
\begin{figure}[!t]
  \begin{centering}
    {\scriptsize
      \fbox{
        \begin{minipage}{6in} 
          \begin{tabbing}
            {\textsc{On-the-fly Taxonomic Relation Identification}~~~~~~~~~~~~~~~~~~~~~~~~~~~~}\\
            \qquad {\textsc{Input}}: A concept pair ($X$, $Y$) \\
            \qquad {\textsc{Output}}: Relation of $X$ and $Y$ \\
            \\
            \qquad If $X$ is a {\em non-Wikipedia concept} then \\
            \qquad \qquad $X \leftarrow \texttt{findReplacement}(X, Y)$ \\
            %\qquad End if \\
            \qquad If $Y$ is a {\em non-Wikipedia concept} then \\
            \qquad \qquad $Y \leftarrow \texttt{findReplacement}(Y, X)$ \\
            %\qquad End if \\
            \\     
            \qquad Building Wikipedia-article representation $F(X)$ and $F(Y)$ \\
            \qquad $R = \texttt{classifyTaxonomicRelation}(F(X),F(Y))$ \\
            \qquad $R^* = \texttt{inference}(X,Y,R)$ \\
            \\
            \qquad \textsc{Return}: $R^*$; \\
          \end{tabbing}
        \end{minipage}
      }
  }
\end{centering}
\caption{On-the-fly taxonomic relation identification algorithm.}
\label{fig:rel-know-iden-alg}
\end{figure}
}


\ignore{In our work, we use Wikipedia as the main source of background
  knowledge used to recognize concept relations.}

Functions \texttt{extractRelatedConcepts} and \texttt{doInference} are
described in details in Sec. \ref{sec:inference}.

The learning component (A) trains a local classifier $\mathcal{R}$ for
the taxonomic relation identification, and is described in Sec. \ref{sec:learning}.

Although most commonly used concepts can be found in Wikipedia, there
is still a need to cover the {\em non-Wikipedia concepts} to improve
the coverage of the algorithm. Briefly, function
\texttt{findReplacement}($X$, $Y$) takes a {\em non-Wikipedia concept}
$X$ and a supporting concept $Y$ as its input. The function searches
the web to find a {\em Wikipedia concept} $X'$ which is in the same
semantic class of $X$ to be its replacement. Our method was motivated
by \cite{1321585}.  In our work, we use the Yahoo! Web Search
APIs\footnote{http://developer.yahoo.com/search/web/} to search for
list structures in web documents such as ``... $\left < delimiter
\right >$ c$_a$ $\left < delimiter \right >$ c$_b$ $\left < delimiter
\right >$ c$_c$ $\left < delimiter \right >$ ...'' ($X$ and $Y$ are
among c$_a$, c$_b$, c$_c$, ...). For text snippets that contain the
patterns of interest, we extract $c_a$, $c_b$, etc. as replacement
candidates. To reduce noise, we constrain the list structure to
contain at least 4 concepts that are no longer than 20 characters
each. The candidates are ranked based on their occurrence
frequency. The top candidate in the Wikipedia concept space is used as
replacement.

%%% Local Variables: 
%%% mode: latex
%%% TeX-master: "jupiter"
%%% End: 
