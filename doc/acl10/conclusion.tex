We study an important component of textual inference--identifying
identifying taxonomic relations between pairs of concepts.  We argued
and provided experimental evidence that static structured knowledge
bases cannot support this task well enough and developed a novel
approach for this problem.  We developed TREI, an algorithm that fuses
information from existing knowledge sources and uses machine learning
and a constrained optimization technique to get around the noise and
the level of uncertainty inherent in these resources. The experimental
study shows that our system is significantly better than other systems
that use existing well-known structured resources, showing that our
approach generalizes well across semantic classes and handles well
concepts that are not mentioned in Wikipedia. The key lesson from the
success of our approach has to do with our combined learning and
global inference approach. Furthermore, we demonstrate an effective
approach to leveraging existing knowledge bases for this inference
process. Our future research will include an evaluation of TREI in the
context a textual inference application. \ignore{Our machine learning
  approach makes use of Wikipedia resource, but views it as an open
  and noisy resource; this is augmented with a novel use of a
  constraint-based inference model that allows us to make these
  decisions more robust.} \ignore{The key technical step needed to
  improve our method further is to better generate concepts that are
  related to the target concepts, so that our global inference method
  becomes even more effective.}

%%% Local Variables:
%%% mode: latex
%%% TeX-master: "jupiter"
%%% End:
