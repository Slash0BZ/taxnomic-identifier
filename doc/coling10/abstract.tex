Determining whether two concepts in text have an ancestor relation
(e.g. {\em Toyota} and {\em car}) or a sibling relation (e.g. {\em
  Toyota} and {\em Honda}) is often essential to support many tasks in
computational linguistics. Significant effort has been devoted to
developing structured knowledge bases that could potentially support
these tasks. But these resources usually suffer from noise and
uncertainty, making their use as background knowledge difficult. We
argue that, given two concepts, the problem of determining what
taxonomic relation holds between them should be addressed in a machine
learning fashion. More importantly, we leverage an existing knowledge
base to enforce relational constraints among concepts in a global
inference process to accurately identify taxonomic
relations. Experiments show that our approach significantly
outperforms other systems built upon existing well-known structured
knowledge bases.

\ignore{Determining whether two concepts in text have an ancestor relation or
a sibling relation is often essential to support textual inferences
such as identifying that {\em ``Jim drives a Honda"} contradicts {\em
  ``Jim drives a Toyota"} but both imply that {\em ``Jim drives a
  Japanese car"}.  Significant effort has been devoted to developing
structured knowledge bases that could potentially support these tasks,
but these lack sufficient coverage and, more significantly, from noise
and unavoidable uncertainty, making their use in textual inference
difficult.  We argue that, given two concepts, the problem of
determining what taxonomic relation holds between them should be
addressed by fusing information from multiple knowledge sources both
structured and unstructured.  We present an approach that makes use of
machine learning and a constrained optimization based decision
algorithm to combine information from structured sources and
unstructured text, and show that it significantly improves taxonomic
relation identification relative to existing structured knowledge
sources.}




\ignore{It was argued that in textual inference tasks, many inferences are
largely compositional and depend on the capability of models to
identify basic relations between entities, noun phrases, etc. While it
is known that several knowledge bases were built to support these
tasks, it is also known that these resources are fixed and usually
restrict the models to a set of rules or hierarchical structures
defining concept relations. In this paper, we present a novel approach
that can identify {\em taxonomic relations} between a pair of concepts
on-the-fly. Especially, we present our innovative constraint-based
framework that takes advantage of related concepts extracted from
other knowledge bases to improve our local prediction of the pairwise
taxonomic relation identification. Our experiments exhibit very large
improvement over methods using existing knowledge sources.}

\ignore{ We present a novel approach to identifying relations between
  a pair of concepts. We focus on identifying relations that are
  essential to support textual inference: determining whether two
  concepts have an ancestor relation, a sibling relation, or no
  relation. We develop a machine learning-based approach that makes
  use of Wikipedia as a main source for background knowledge, but we
  also propose an effective approach of searching the Web to improve
  the coverage of our method and support inference between concepts
  not present in Wikipedia. Our key innovation is that, in order to
  accurately determine the relations between concepts $C_1$ and $C_2$,
  we consider an automatically generated collection of related
  concepts, evaluate all pairwise relations, and use constraint-based
  inference to force them to cohere, thus improving the local
  prediction of the pairwise relation identification. We demonstrate
  that the inference technique significantly enhances the local
  prediction methods and consequently exhibit very large improvements
  over methods using existing knowledge sources.  }

%%% Local Variables:
%%% mode: latex
%%% TeX-master: "jupiter"
%%% End:
