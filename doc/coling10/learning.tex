\ignore{ Given two input concepts ($X$, $Y$), we first disambiguate
  them to get lists of relevant Wikipedia articles. We then extract
  features of disambiguated concepts and train a supervised
  multi-class classifier to classify relations between them.  }

In this section, we describe our local taxonomic relation classifier.

For each input concept in the input pair, we first build its
Wikipedia-article representation which is a list of top relevant
articles in Wikipedia retrieved by a local search
engine\footnote{E.g. http://lucene.apache.org/}. To do this, we use
the following procedure: (1) Using both concepts to make a query
(e.g. {\em ``George W. Bush''} AND {\em ``Bill Clinton''}) to search
in Wikipedia ; (2) Extracting important keywords in the titles and
categories of the retrieved articles using TF-IDF (e.g. {\em
  president}, {\em politician}); (3) Combining each input concept with
the extracted keywords (e.g. {\em ``George W. Bush''} AND {\em
  ``president''} AND {\em ``politician''}) to create a final query
used to search for the concept's relevant articles in Wikipedia. This
procedure was motivated by the assumption that real world applications
usually have a need to know the taxonomic relation between two
concepts which are related and in the same sense. For example, if the
input pair is ({\em George W. Bush}, {\em Ford}), it's more
likely that concept {\em Ford} refers to the former president of
the United States, {\em Gerald Ford}, than the founder of Ford Motor
Company, {\em Henry Ford}.

\begin{table*}[!t]
  \tiny
  \centering
  \begin{tabular}{|p{0.5in}|p{3.7in}|p{1.6in}|}
    \hline
    {\bf Title/Concept} & {\bf Text} & {\bf Categories} \\
    \hline
    \hline
    President of the United States & \textit{The President of the United States is the head of state and head of government of the United States and is the highest political official in the United States by influence and recognition. The President leads the executive branch of the federal government and is one of only two elected members of the executive branch...} & \textit{Presidents of the United States, Presidency of the United States} \\
    \hline
    George W. Bush & \textit{George Walker Bush; born July 6, 1946) served as the 43rd President of the United States from 2001 to 2009. He was the 46th Governor of Texas from 1995 to 2000 before being sworn in as President on January 20, 2001...} & \textit{Children of Presidents of the United States, Governors of Texas, Presidents of the United States, Texas Republicans...} \\
    \hline
    Gerald Ford & \textit{Gerald Rudolff Ford (born Leslie Lynch King, Jr.) (July 14, 1913 – December 26, 2006) was the 38th President of the United States, serving from 1974 to 1977, and the 40th Vice President of the United States serving from 1973 to 1974.} & \textit{Presidents of the United States, Vice Presidents of the United States, Republican Party (United States) presidential nominees...} \\
    \hline
  \end{tabular}
  \caption{Examples of texts and categories of Wikipedia articles.}
  \label{tab:snippets}
\end{table*}

After having Wikipedia-article representation for each concept, we
extract learning features of two input concepts from the articles in
the representations. All these features are used to train a local
taxonomic relation classifier. It is worth noting that a Wikipedia
page usually consists of a title (i.e. the concept), a body text, and
a list of categories to which the page belongs. Table
\ref{tab:snippets} shows some Wikipedia articles. From now on, we use
{\em the titles of $X$}, {\em the texts of $X$}, and {\em the
  categories of $X$} to refer to the titles, texts, and categories of
the associated articles in the representation of $X$.  The learning
features extracted for input pair ($X$,$Y$) are described below.

\ignore{
  \subsection{Concept Disambiguation}
  \label{sec:conc-disamb}

  \ignore{In this paper, we address the problem of identifying
    relation between concepts.  Our problem is different than other
    previous work on relation extraction. Most of traditional relation
    extraction tasks often focus on extracting all possible entity
    pairs that follow some pre-defined relations (e.g. LocatedIn,
    BornInYear, AuthorOf, etc.) in big corpora
    \cite{banko-etzioni:2008:ACLMain}. Some other previous work
    extracting all possible entity pairs and relations from big
    corpora with a list of same class entities as the start point
    \cite{davidov-rappoport:2008:ACLMain2}. Another direction of
    relation extraction is to extract all possible instances of a
    given class, or in the same class of given instances
    \cite{WangCohen09,kozareva-riloff-hovy:2008:ACLMain}.}

  \ignore{In this paper, we limit the relations of interest to
    \emph{Ancestor}, \emph{Sibling}, or \emph{None} relations.}

  Given two input concepts ($X$, $Y$), we must first disambiguate
  them. The more accurate the disambiguation is, the more precise the
  relation recognition is. For example, the concept \emph{Ford} in the
  concept pair \emph{(Bush, Ford)} may refer to several concepts, such
  as \emph{Ford Motor Company}, president \emph{Gerald Ford}, or
  \emph{Ford, Wisconsin - USA}. Our approach is motivated by observing
  that the concept \emph{Bush} can be used to disambiguate \emph{Ford}
  and vice versa. Ultimately, we can place articles about \emph{George
    W. Bush} and \emph{Gerald Ford} the top of a retrieved articles
  list for the two input concepts \emph{Bush} and \emph{Ford}.

  Figure \ref{fig:entity-disamb} shows the concept disambiguation
  algorithm.

  \begin{figure}[!t]
    \begin{centering}
      {\scriptsize \fbox{
        \begin{minipage}{6in} 
          \begin{tabbing}
            {\textsc{Concept Disambiguation Algorithm}} \\
            \qquad {\textsc{Input}}: Concept pair ($X$, $Y$); Number of top tokens $N$ ~~~~~~~ \\
            \qquad {\textsc{Output}}: Two list of relevant articles for $X$ and $Y$. \\
            \\
            \qquad Query $q \leftarrow $ Concatenating $X$ and $Y$; \\
            \qquad $\mathcal{L} = \texttt{ESA}(q)$; \\ %// return a list of relevant articles to $X$ and $Y$ \\
            \qquad $\mathcal{C} = \texttt{getCategories}(\mathcal{L})$; \\
            \qquad $\mathcal{T} = \texttt{tokenize}(\mathcal{C})$; \\
            \qquad $\mathcal{T}_{top} = \texttt{topTfidf}(\mathcal{T}, N)$; \\ %// pick top $N$ tokens\\
            \\
            \qquad Query $q_X \leftarrow $ Concatenating $X$ and tokens in $\mathcal{T}_{top}$; \\
            \qquad Query $q_Y \leftarrow $ Concatenating $Y$ and tokens in $\mathcal{T}_{top}$; \\
            \qquad $\mathcal{A}_X = \texttt{search}(q_X)$; \\
            \qquad $\mathcal{A}_Y = \texttt{search}(q_Y)$; \\
            \\
            \qquad \textsc{Return}: $\mathcal{A}_X$ and $\mathcal{A}_Y$; \\
          \end{tabbing}
        \end{minipage}
      } }
  \end{centering}
  \caption{Concept disambiguation algorithm. The algorithm takes two
    input concepts, disambiguates them and returns a list of relevant
    Wikipedia articles for each.}
  \label{fig:entity-disamb}
  \end{figure}

  Function \texttt{ESA}({\em query}) retrieves the most semantically
  relevant articles in Wikipedia to {\em query}.  We use Explicit
  Semantic Analysis (ESA) \cite{GabrilovichMa07} to retrieve
  semantically relevant articles in Wikipedia for two given
  concepts. For all articles in the relevant list, we extract and keep
  their categories by the function \texttt{getCategories}({\em
    list}). All categories are then tokenized and weighted using the
  TF-IDF score. Only top $N$ tokens, which are different from the two
  input concepts, are considered. Each input concept is then
  concatenated with the top weighted tokens to create a new query
  that, in turn, is the input for the \texttt{search}({\em query})
  function. Our search function returns relevant articles in Wikipedia
  for the input {\em query}. The lists of relevant articles are later
  used to provide features for the local relation classifier. If a
  returned list is empty, its corresponding concept is considered as a
  {\em non-Wikipedia concept}.  }


%\subsection{Learning Relations}
%\label{sec:learner}

\ignore{ We first extract expressive features associated with the two
  input concept pairs. These features are necessarily independent of
  the semantic class of the given concepts so that our classifier can
  generalize well. Recall that, after concept disambiguation, each
  input concept is associated with a list of relevant articles in
  Wikipedia. In other words, from these articles, an input concept is
  represented with several Wikipedia titles, texts, and categories.
  From now on, we use {\em the titles of concept $X$} to refer to the
  titles of the articles associated with concept $X$; similarly for
  {\em the texts of concept $X$}, and {\em the categories of concept
    $X$}. As a learning algorithm, we use a regularized averaged
  Perceptron \cite{FreundSc99}.  }

\ignore{The features selected for our learning problem include the
  words used in the associated articles, the association information
  between two concepts, and the overlap ratios of important
  information between the concepts and the articles. We describe these
  features below.}

%\subsubsection{Bags of Words}

\ignore{{\bf Bags of words}: One of the most important features in
  recognizing relations between concepts are the words used in
  associated articles. An article in Wikipedia typically contains
  three main components: title, text, and categories. The title
  distinguishes a concept from other concepts in Wikipedia; the text
  describes the concept; and the categories classify the concept into
  one or more concept groups which can be further categorized. To
  collect the categories for a concept, we take the categories of its
  associated articles and go up $K$ levels in the Wikipedia category
  system. In our experiments, we use abstracts of Wikipedia articles
  instead of whole texts.}

{\bf Bags-of-words Similarity:} We use cosine similarity metric to
measure the degree of similarity between bags of words. We define four
bags-of-words similarity features associated with any two given
concepts $X$ and $Y$: (1) the degree of similarity between the texts
of $X$ and the categories of $Y$, (2) the similarity between the
categories of $X$ and the texts of $Y$, (3) the similarity between the
texts of $X$ and the texts of $Y$, and (4) the similarity between the
categories of $X$ and the categories of $Y$. To collect categories of
a concept, we take the categories of its associated articles and go up
$K$ levels in the Wikipedia category system. In our experiments, we
use abstracts of Wikipedia articles instead of whole texts.

\ignore{ Table \ref{tab:snippets} shows three related concepts and
  their associated articles in Wikipedia\footnote{Note that each
    concept can be associated with more than one article in Wikipedia
    depending on the list of relevant articles returned by the concept
    disambiguation algorithm.}. We see the overlap in the words used
  in the title, the text and the categories of these three
  concepts. The overlap in word usage determines the degree of
  similarity of concepts holding relations of interest.  }

\ignore{ There are several ways to calculate the degree of similarity
  between two text fragments \cite{mohler-mihalcea:2009:EACL}. We use
  the cosine similarity metric. The following equation shows the
  cosine similarity metric applied to the term vectors $T_1$ and $T_2$
  of two text fragments.

  \begin{equation}
    \label{eq:1}
    Sim(T_1, T_2) = \frac{T_1 \cdot T_2}{\left\| T_1 \right\| \left\| T_2 \right\|} = \frac{\sum_{i=1}^{N}x_iy_i}{\sqrt{\sum_{j=1}^{N}x^2_j} \sqrt{\sum_{k=1}^{N}y^2_k}}  \notag
  \end{equation}

  where $N$ is the vocabulary size, and $x_i$, $y_i$ are two indicator
  values in $T_1$ and $T_2$, respectively.  }

%\subsubsection{Association Information}

{\bf Association Information:} Intuitively, association information is
the overlap information that is shared by any two concepts. We capture
this feature by the pointwise mutual information which quantifies the
discrepancy between the probability of two concepts appearing together
versus the probability of each concept appearing independently. The
pointwise mutual information for two concepts $X$ and $Y$ is defined
as following.
%
\begin{equation}
  \label{eq:2}
  % PMI(X, Y) = log \frac{p(X, Y)}{p(X)p(Y)} = log \frac{\frac{f(X,Y)}{N}}{\frac{f(X)}{N} \frac{f(Y)}{N}} = log\frac{N f(X,Y)}{f(X)f(Y)} \notag
  PMI(X, Y) = log \frac{p(X, Y)}{p(X)p(Y)} = log\frac{N f(X,Y)}{f(X)f(Y)} \notag
\end{equation}
%
where $N$ is the total number of Wikipedia articles, and $f(.)$ is the
function which counts the number of appearance of its argument.

% \begin{equation}
% \label{eq:2}
%   PMI(x, y) = log \frac{p(x, y)}{p(x)p(y)} = log \frac{\frac{C(x,y)}{N}}{\frac{C(x)}{N}\frac{C(y)}{N}} = log\frac{N \times C(x,y)}{C(x)C(y)}
% \end{equation}

\ignore{ where $X$ and $Y$ are two input entities, $N$ is the total
  number of documents in Wikipedia, and $f(.)$ is a function returning
  document frequency. We measure association information at the
  document level.  }

%\subsubsection{Overlap Ratio}

{\bf Overlap Ratios:} The overlap ratio features captures the fact
that the titles of a concept usually overlap with the categories of
its descendants.  \ignore{For examples, the title (and also the
  concept itself) {\em Presidents of the United States} overlap with
  one of the categories of the concepts {\em George W. Bush} and {\em
    Gerald Ford} (see Tab. \ref{tab:snippets}.)}  We measure this
overlap as the ratio of {\em common phrases} used in the titles of one
concept and the categories of the other concept. In our context, a
phrase is considered to be a {\em common phrase} if it appears in the
titles of one concept and the categories of the other concept and is
one of the following: (1) the {\em whole string} of a category, or (2)
the {\em head} in its root form of a category, or (3) the {\em
  post-modifier} of a category. We use the Noun Group Parser from
\cite{suchanek2007WWW} to extract the {\em head} and {\em
  post-modifier} from a category. For example, one of the categories
of an article about \emph{Chicago} is \emph{Cities in Illinois}. This
category can be parsed into a head in its root form \emph{City}, and a
post-modifier \emph{Illinois}. Given concept pair \emph{(City,
  Chicago)}, we observe that \emph{City} matches the head of the
category {\em Cities in Illinois} of concept {\em Chicago}. Thus, we
have a strong indication that {\em Chicago} is a child of {\em City}.

% \begin{itemize}
% \item the {\em whole string} of a category of $Y$, or
% \item the {\em head} in its root form of a category of $Y$, or
% \item the {\em post-modifier} of a category of $Y$.
% \end{itemize}

\ignore{ We measure the overlap ratio between the titles of concept
  $X$ and the categories of concept $Y$ up to $K$ levels in the
  Wikipedia category system by using the Jaccard similarity
  coefficient.

\begin{equation}
  \label{eq:3}
  \sigma_{ttl}^K(X,Y) = \frac{\left| L_X \cap \mathcal{P}_Y^K \right|}{\left| L_X \cup \mathcal{P}_Y^K \right|} \notag
\end{equation}

where $L_X$ is the set of the titles of $X$, and $\mathcal{P}_Y$ is
the union of the category strings, the heads of the categories, and
the post-modifiers of the categories of $Y$. Similarly, we also use
the ratio $\sigma_{ttl}^K(Y,X)$ as one of our features.
}

We also use a feature that captures the overlap ratio of {\em common
  phrases} between the categories of two input concepts. For this
feature, we do not use the {\em post-modifier} of the categories.  We
use Jaccard similarity coefficient to measure these overlaps ratios.

\ignore{
The Jaccard similarity coefficient for overlap ratio
between the categories of two concepts $X$ and $Y$ is shown in the
following equation.

\begin{equation}
  \label{eq:4}
  \sigma_{cat}^K(X,Y) = \frac{\left| \mathcal{Q}_X^K \cap \mathcal{Q}_Y^K \right|}{\left| \mathcal{Q}_X^K \cup \mathcal{Q}_Y^K \right|} \notag
\end{equation}

where $\mathcal{Q}_{(.)}^K$ is the union of the category strings and
the heads of the categories. The categories are collected by going up
$K$ levels in the Wikipedia category system.

All of the features described above are used to train a local
multi-class classifier to recognize relations between concepts.
}

%%% Local Variables: 
%%% mode: latex
%%% TeX-master: "jupiter"
%%% End: 
